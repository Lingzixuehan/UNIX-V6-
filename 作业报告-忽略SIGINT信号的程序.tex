\documentclass[12pt,a4paper]{ctexart}
\usepackage{geometry}
\usepackage{listings}
\usepackage{xcolor}
\usepackage{graphicx}
\usepackage{hyperref}
\usepackage{enumitem}
\usepackage{tikz}
\usetikzlibrary{shapes.geometric, arrows}

\geometry{left=2.5cm,right=2.5cm,top=2.5cm,bottom=2.5cm}

% 代码高亮设置
\lstset{
    language=C,
    basicstyle=\ttfamily\small,
    keywordstyle=\color{blue}\bfseries,
    commentstyle=\color{gray}\itshape,
    stringstyle=\color{red},
    numbers=left,
    numberstyle=\tiny\color{gray},
    stepnumber=1,
    numbersep=8pt,
    backgroundcolor=\color{gray!10},
    showspaces=false,
    showstringspaces=false,
    showtabs=false,
    frame=single,
    tabsize=4,
    captionpos=b,
    breaklines=true,
    breakatwhitespace=false,
    escapeinside={\%*}{*)},
    xleftmargin=2em,
    xrightmargin=2em,
    aboveskip=1em,
}

\hypersetup{
    colorlinks=true,
    linkcolor=blue,
    filecolor=magenta,
    urlcolor=cyan,
}

\title{\textbf{UNIX V6++ 信号处理实验报告(一)\\忽略 SIGINT 信号的不死进程}}
\author{}
\date{\today}

\begin{document}

\maketitle
\tableofcontents
\newpage

\section{仓库说明}

本仓库是 \textbf{UNIX V6++} 操作系统的实现代码,这是一个基于经典 UNIX Version 6 的教学操作系统项目。系统包含完整的内核实现,包括:

\begin{itemize}
    \item \textbf{进程管理}:进程调度、进程间通信
    \item \textbf{内存管理}:分页机制、内存交换
    \item \textbf{文件系统}:inode、目录管理
    \item \textbf{信号机制}:支持 32 种标准 UNIX 信号
    \item \textbf{设备驱动}:键盘、鼠标、磁盘等
\end{itemize}

\section{实现原理}

\subsection{信号机制基础}

在 UNIX 系统中,按下 \textbf{Ctrl+C} 会向前台进程发送 \textbf{SIGINT (信号 2)} 信号,默认行为是终止进程。但该信号可以被捕获和处理,程序可以选择忽略或自定义处理。

\subsection{关键技术点}

\subsubsection{可捕获的信号}

\begin{itemize}
    \item SIGINT (2) - Ctrl+C
    \item SIGTERM (15) - 终止请求
    \item SIGUSR1 (10) - 用户自定义信号 1
    \item SIGUSR2 (12) - 用户自定义信号 2
\end{itemize}

\subsubsection{不可捕获的信号}

\begin{itemize}
    \item \textbf{SIGKILL (9)} - 强制杀死
    \item \textbf{SIGSTOP (19)} - 强制暂停
\end{itemize}

\section{程序实现}

\subsection{核心代码}

文件位置:\texttt{src/program/immortal.c}

\begin{lstlisting}[caption={immortal.c - 核心实现}]
#include <stdio.h>
#include <sys.h>

// SIGINT 信号处理函数 - 必须接受 int 参数
void sigint_handler(int signo)
{
    if (signo == SIGINT)
    {
        printf("\n[!] Caught SIGINT (Ctrl+C), but I refuse to die!\n");
        // 重要:在 UNIX V6 中,signal() 是一次性的
        // 必须在处理函数中重新注册,否则第二次 Ctrl+C 会终止进程
        signal(SIGINT, sigint_handler);
    }
}

int main1(int argc, char* argv[])
{
    // 注册信号处理函数,捕获 SIGINT 但不退出
    if (signal(SIGINT, sigint_handler) == -1)
    {
        printf("Signal registration failed!\n");
        return -1;
    }

    printf("Immortal Process (PID: %d)\n", getpid());
    printf("Press Ctrl+C to test - I won't die!\n");
    printf("Use 'kill -9 %d' to kill me.\n", getpid());
    printf("Starting...\n\n");

    // 主循环:持续运行
    while(1)
    {
        sleep(10);
        printf("Still alive (PID: %d)\n", getpid());
    }

    return 0;
}
\end{lstlisting}

\subsection{关键要点}

\subsubsection{要点 1:信号处理函数签名}

在 UNIX V6++ 系统中,信号处理函数\textbf{必须接受一个 int 参数}:

\begin{lstlisting}[caption={正确的信号处理函数签名}]
// 正确
void sigint_handler(int signo)

// 错误(会导致信号无法正确捕获)
void sigint_handler()
\end{lstlisting}

这个参数表示接收到的信号编号,即使不使用也必须声明。

\subsubsection{要点 2:signal() 是一次性的(最关键!)}

UNIX V6 的 \texttt{signal()} 系统调用是\textbf{一次性的}:

\begin{lstlisting}[caption={Process.cpp 中的信号处理代码(第 481 行)}]
// 在 src/proc/Process.cpp 的 PSig() 函数中:
u.u_signal[signal] = 0;  // 信号处理后自动重置为 0(默认行为)
\end{lstlisting}

\textbf{后果:}
\begin{itemize}
    \item 第 1 次按 Ctrl+C:✓ 调用信号处理函数
    \item 第 2 次按 Ctrl+C:✗ 进程被终止(因为信号处理已重置为默认)
\end{itemize}

\textbf{解决方案:}

必须在信号处理函数中\textbf{重新注册}信号处理:

\begin{lstlisting}[caption={在信号处理函数中重新注册}]
void sigint_handler(int signo)
{
    printf("Caught signal!\n");
    signal(SIGINT, sigint_handler);  // 必须重新注册!
}
\end{lstlisting}

这是经典 UNIX 系统(V6/V7)的标准行为,现代 UNIX 使用 \texttt{sigaction()} 来避免这个问题。

\subsection{实现方式}

本程序提供了两种实现方式:

\textbf{方式 1:自定义处理函数(已采用)}

\begin{lstlisting}
signal(SIGINT, sigint_handler);
\end{lstlisting}

\begin{itemize}
    \item 捕获 SIGINT 信号
    \item 执行自定义处理函数
    \item 显示提示信息但不退出
\end{itemize}

\textbf{方式 2:完全忽略信号}

\begin{lstlisting}
signal(SIGINT, SIG_IGN);
\end{lstlisting}

\begin{itemize}
    \item 直接忽略 SIGINT 信号
    \item 按 Ctrl+C 没有任何反应
\end{itemize}

\section{构建系统集成}

\subsection{修改 Makefile}

在 \texttt{src/program/Makefile} 中添加编译规则:

\begin{lstlisting}[language=make,caption={Makefile 配置}]
# 1. 添加到目标列表
SHELL_OBJS = ... \
             $(TARGET)\immortal.exe

# 2. 添加编译规则
$(TARGET)\immortal.exe : immortal.c
    $(CC) $(CFLAGS) -I"$(INCLUDE)" -I"$(LIB_INCLUDE)" \
          $< -e _main1 $(V6++LIB) -o $@
    copy $(TARGET)\immortal.exe $(MAKEIMAGEPATH)\$(BIN)\immortal
\end{lstlisting}

\subsection{编译参数说明}

\begin{itemize}
    \item \texttt{-nostdlib -nostartfiles -nostdinc}:裸机编译,不依赖标准库
    \item \texttt{-e \_main1}:指定入口点为 \texttt{main1} 函数(V6++ 规范)
    \item \texttt{\$(V6++LIB)}:链接 V6++ 系统库,提供 signal、printf 等函数
\end{itemize}

\section{运行效果}

\subsection{预期输出}

\begin{verbatim}
Immortal Process (PID: 42)
Press Ctrl+C to test - I won't die!
Use 'kill -9 42' to kill me.
Starting...

Still alive (PID: 42)
Still alive (PID: 42)
^C
[!] Caught SIGINT (Ctrl+C), but I refuse to die!
Still alive (PID: 42)
Still alive (PID: 42)
\end{verbatim}

\subsection{测试结果}

\begin{itemize}
    \item[✓] 按 Ctrl+C:程序捕获信号,显示提示,继续运行
    \item[✓] 使用 \texttt{kill -9 42}:程序被强制终止
\end{itemize}

\subsection{调试经验}

\subsubsection{问题 1:信号处理函数签名错误}

\begin{itemize}
    \item \textbf{现象}:按 Ctrl+C 后程序立即退出
    \item \textbf{原因}:函数定义为 \texttt{void handler()} 而非 \texttt{void handler(int signo)}
    \item \textbf{解决}:参考 sigTest.c 添加 int 参数
\end{itemize}

\subsubsection{问题 2:第二次 Ctrl+C 导致退出(最难排查)}

\begin{itemize}
    \item \textbf{现象}:第一次 Ctrl+C 被捕获,第二次直接退出
    \item \textbf{原因}:UNIX V6 的 signal() 是一次性的(Process.cpp:481)
    \item \textbf{解决}:在信号处理函数中调用 \texttt{signal(SIGINT, sigint\_handler)} 重新注册
    \item \textbf{教训}:经典 UNIX 与现代 UNIX 的信号机制差异
\end{itemize}

\subsubsection{问题 3:输出混乱}

\begin{itemize}
    \item \textbf{现象}:终端显示大量调试信息和进程状态
    \item \textbf{原因}:输出过于频繁触发系统调试
    \item \textbf{解决}:降低输出频率(1秒 → 10秒)
\end{itemize}

\section{技术总结}

\subsection{关键知识点}

\begin{enumerate}
    \item \textbf{信号处理机制}:\texttt{signal(int sig, void (*handler)())} 系统调用
    \item \textbf{信号类型}:可捕获信号 vs 不可捕获信号
    \item \textbf{经典 UNIX 特性}:signal() 的一次性行为(需要重新注册)
    \item \textbf{进程控制}:getpid()、sleep()、while 循环
    \item \textbf{裸机编程}:无标准库环境下的系统编程
    \item \textbf{内核源码分析}:通过阅读 Process.cpp 理解信号处理的内部实现
\end{enumerate}

\subsection{实际应用场景}

\begin{itemize}
    \item \textbf{服务器守护进程}:忽略 SIGINT,避免误操作关闭
    \item \textbf{数据库系统}:捕获信号,优雅关闭(保存数据后退出)
    \item \textbf{长时间任务}:防止用户误按 Ctrl+C 中断任务
    \item \textbf{调试工具}:捕获所有信号用于调试分析
\end{itemize}

\section{补充说明}

本次作业还顺带修复了 Makefile 中 \texttt{kill\_child.exe} 缺失的编译规则,确保所有程序都能正确编译。

\vspace{1em}
\noindent\rule{\textwidth}{0.4pt}

\begin{center}
\textbf{开发环境:} UNIX V6++ 操作系统 \\
\textbf{编译工具:} gcc (裸机模式) \\
\textbf{测试状态:} ✓ 通过编译,已集成到构建系统
\end{center}

\end{document}
