\documentclass[12pt,a4paper]{ctexart}
\usepackage{geometry}
\usepackage{listings}
\usepackage{xcolor}
\usepackage{graphicx}
\usepackage{hyperref}
\usepackage{enumitem}
\usepackage{tikz}
\usepackage{booktabs}
\usetikzlibrary{shapes.geometric, arrows, positioning}

\geometry{left=2.5cm,right=2.5cm,top=2.5cm,bottom=2.5cm}

% 代码高亮设置
\lstset{
    language=C,
    basicstyle=\ttfamily\small,
    keywordstyle=\color{blue}\bfseries,
    commentstyle=\color{gray}\itshape,
    stringstyle=\color{red},
    numbers=left,
    numberstyle=\tiny\color{gray},
    stepnumber=1,
    numbersep=8pt,
    backgroundcolor=\color{gray!10},
    showspaces=false,
    showstringspaces=false,
    showtabs=false,
    frame=single,
    tabsize=4,
    captionpos=b,
    breaklines=true,
    breakatwhitespace=false,
    escapeinside={\%*}{*)},
    xleftmargin=2em,
    xrightmargin=2em,
    aboveskip=1em,
}

\hypersetup{
    colorlinks=true,
    linkcolor=blue,
    filecolor=magenta,
    urlcolor=cyan,
}

\title{\textbf{UNIX V6++ 信号处理实验报告(二)\\父进程使用 SIGKILL 杀死子进程}}
\author{}
\date{\today}

\begin{document}

\maketitle
\tableofcontents
\newpage

\section{程序功能说明}

本程序演示了父进程如何使用 \texttt{kill()} 系统调用发送 \textbf{SIGKILL} 信号来强制终止子进程。

\subsection{功能要求}

\begin{enumerate}
    \item[✓] 父进程创建一个子进程
    \item[✓] fork 返回后,子进程进入无限循环
    \item[✓] 父进程执行 kill 系统调用,用 SIGKILL 信号杀死子进程
\end{enumerate}

\section{程序实现}

\subsection{核心代码}

文件位置:\texttt{src/program/kill\_child.c}

\begin{lstlisting}[caption={kill\_child.c - 完整实现}]
#include <stdio.h>
#include <sys.h>

int main1(int argc, char* argv[])
{
    int pid = fork();

    if (pid == 0) /* 子进程 */
    {
        printf("[Child] PID: %d - Entering infinite loop...\n", getpid());
        printf("[Child] Waiting to be killed by parent.\n");

        // 子进程无限循环,等待被杀死
        while (1)
        {
            // 空循环
        }
    }
    else if (pid > 0) /* 父进程 */
    {
        printf("[Parent] PID: %d - Created child with PID: %d\n",
               getpid(), pid);

        // 等待 2 秒,确保子进程已经开始运行
        printf("[Parent] Waiting 2 seconds before killing child...\n");
        sleep(2);

        // 发送 SIGKILL 信号杀死子进程
        printf("[Parent] Sending SIGKILL to child (PID: %d)...\n", pid);
        int result = kill(pid, SIGKILL);

        if (result == -1)
        {
            printf("[Parent] ERROR: Failed to send SIGKILL!\n");
            return -1;
        }
        else
        {
            printf("[Parent] SIGKILL sent successfully.\n");
        }

        // 等待子进程结束并获取退出状态
        int status;
        int wpid = wait(&status);
        printf("[Parent] Child process (PID: %d) has been terminated.\n",
               wpid);
        printf("[Parent] Child exit status: %d\n", status);
        printf("[Parent] Done.\n");
    }
    else /* fork 失败 */
    {
        printf("ERROR: Fork failed!\n");
        return -1;
    }

    return 0;
}
\end{lstlisting}

\subsection{程序流程图}

\begin{center}
\begin{tikzpicture}[
    node distance=1.5cm,
    startstop/.style={rectangle, rounded corners, minimum width=3cm, minimum height=1cm, text centered, draw=black, fill=red!20},
    process/.style={rectangle, minimum width=3cm, minimum height=1cm, text centered, draw=black, fill=blue!20},
    decision/.style={diamond, minimum width=3cm, minimum height=1cm, text centered, draw=black, fill=green!20},
    arrow/.style={thick,->,>=stealth}
]

\node (start) [startstop] {主进程启动};
\node (fork) [process, below of=start] {fork()};
\node (parent) [process, below left of=fork, xshift=-2cm, yshift=-1cm] {父进程};
\node (child) [process, below right of=fork, xshift=2cm, yshift=-1cm] {子进程};

\node (p1) [process, below of=parent] {打印信息};
\node (p2) [process, below of=p1] {sleep(2)};
\node (p3) [process, below of=p2] {kill(pid, SIGKILL)};
\node (p4) [process, below of=p3] {wait(\&status)};
\node (p5) [startstop, below of=p4] {结束};

\node (c1) [process, below of=child] {打印 PID};
\node (c2) [process, below of=c1] {while(1) 循环};
\node (c3) [startstop, below of=c2] {被杀死};

\draw [arrow] (start) -- (fork);
\draw [arrow] (fork) -| (parent);
\draw [arrow] (fork) -| (child);
\draw [arrow] (parent) -- (p1);
\draw [arrow] (p1) -- (p2);
\draw [arrow] (p2) -- (p3);
\draw [arrow] (p3) -- (p4);
\draw [arrow] (p4) -- (p5);
\draw [arrow] (child) -- (c1);
\draw [arrow] (c1) -- (c2);
\draw [arrow] (c2) -- (c3);
\draw [arrow, dashed, red] (p3) -- (c3);

\end{tikzpicture}
\end{center}

\section{关键技术点}

\subsection{系统调用}

\subsubsection{fork() - 创建子进程}

\begin{lstlisting}
int pid = fork();
\end{lstlisting}

\textbf{返回值:}
\begin{itemize}
    \item 父进程中:返回子进程的 PID ($>$ 0)
    \item 子进程中:返回 0
    \item 失败:返回 -1
\end{itemize}

\subsubsection{kill() - 发送信号}

\begin{lstlisting}
int kill(int pid, int signal);
\end{lstlisting}

\textbf{参数:}
\begin{itemize}
    \item \texttt{pid}:目标进程的 PID
    \item \texttt{signal}:要发送的信号编号(这里是 SIGKILL)
\end{itemize}

\textbf{返回值:}
\begin{itemize}
    \item 成功:0
    \item 失败:-1
\end{itemize}

\subsubsection{wait() - 等待子进程结束}

\begin{lstlisting}
int wait(int* status);
\end{lstlisting}

\begin{itemize}
    \item \textbf{功能}:阻塞等待任一子进程结束,回收其资源
    \item \textbf{返回值}:已结束子进程的 PID
    \item \textbf{参数}:\texttt{status} 保存子进程的退出状态
\end{itemize}

\subsection{SIGKILL 信号特性}

\begin{table}[h]
\centering
\begin{tabular}{ll}
\toprule
\textbf{特性} & \textbf{说明} \\
\midrule
信号编号 & 9 (在 sys.h 中定义) \\
能否被捕获 & ✗ 不能 \\
能否被忽略 & ✗ 不能 \\
能否被阻塞 & ✗ 不能 \\
用途 & 强制终止进程(终极杀手) \\
\bottomrule
\end{tabular}
\caption{SIGKILL 信号特性}
\end{table}

\subsubsection{SIGKILL vs SIGINT 对比}

\begin{lstlisting}[caption={信号捕获对比}]
// SIGINT (Ctrl+C) - 可以被捕获
signal(SIGINT, my_handler);  // 可以捕获并自定义处理

// SIGKILL (kill -9) - 无法被捕获
signal(SIGKILL, my_handler); // 系统不允许,SIGKILL 必定杀死进程
\end{lstlisting}

\section{运行效果}

\subsection{预期输出}

\begin{verbatim}
[Parent] PID: 5 - Created child with PID: 6
[Child] PID: 6 - Entering infinite loop...
[Child] Waiting to be killed by parent.
[Parent] Waiting 2 seconds before killing child...
[Parent] Sending SIGKILL to child (PID: 6)...
[Parent] SIGKILL sent successfully.
[Parent] Child process (PID: 6) has been terminated.
[Parent] Child exit status: 9
[Parent] Done.
\end{verbatim}

\subsection{执行步骤}

\begin{enumerate}
    \item \textbf{主进程调用 fork()}
    \begin{itemize}
        \item 创建子进程(PID: 6)
        \item 父进程得到子进程 PID
    \end{itemize}

    \item \textbf{子进程执行}
    \begin{itemize}
        \item 打印自己的 PID
        \item 进入 \texttt{while(1)} 无限循环
    \end{itemize}

    \item \textbf{父进程执行}
    \begin{itemize}
        \item 打印子进程 PID
        \item 休眠 2 秒(确保子进程开始运行)
        \item 调用 \texttt{kill(6, SIGKILL)} 发送 SIGKILL 信号
    \end{itemize}

    \item \textbf{子进程被杀死}
    \begin{itemize}
        \item 内核强制终止子进程
        \item 无法执行任何清理代码
    \end{itemize}

    \item \textbf{父进程回收}
    \begin{itemize}
        \item \texttt{wait()} 返回子进程 PID
        \item 获取退出状态(通常为信号编号)
    \end{itemize}
\end{enumerate}

\section{技术对比}

\subsection{与 immortal 程序的对比}

\begin{table}[h]
\centering
\begin{tabular}{cccc}
\toprule
\textbf{程序} & \textbf{信号} & \textbf{能否捕获} & \textbf{能否存活} \\
\midrule
immortal & SIGINT (2) & ✓ 可以 & ✓ 不死 \\
kill\_child & SIGKILL (9) & ✗ 不能 & ✗ 必死 \\
\bottomrule
\end{tabular}
\caption{两个程序的对比}
\end{table}

\subsection{可捕获信号 vs 不可捕获信号}

\textbf{可捕获信号(示例):}
\begin{itemize}
    \item SIGINT (2) - Ctrl+C
    \item SIGTERM (15) - 终止请求
    \item SIGUSR1 (10) - 用户自定义信号 1
    \item SIGUSR2 (12) - 用户自定义信号 2
\end{itemize}

\textbf{不可捕获信号:}
\begin{itemize}
    \item \textbf{SIGKILL (9)} - 强制杀死
    \item \textbf{SIGSTOP (19)} - 强制暂停
\end{itemize}

\section{实际应用场景}

\subsection{SIGKILL 的使用场景}

\subsubsection{场景 1:进程无响应}

进程挂死,不响应 SIGTERM

\begin{lstlisting}[language=bash]
kill -15 1234   # 先尝试友好终止
sleep 5
kill -9 1234    # 如果还活着,强制杀死
\end{lstlisting}

\subsubsection{场景 2:紧急情况}

需要立即终止进程,无法等待清理

\begin{lstlisting}[language=bash]
kill -9 <pid>   # 立即终止
\end{lstlisting}

\subsubsection{场景 3:系统管理}

清理僵尸进程或失控进程

\subsection{最佳实践}

\textbf{推荐的进程终止流程:}

\begin{lstlisting}[caption={优雅终止进程}]
// 1. 先发送 SIGTERM(温柔终止)
kill(pid, SIGTERM);
sleep(5);

// 2. 检查进程是否还存在
if (process_exists(pid)) {
    // 3. 如果还存在,使用 SIGKILL(强制终止)
    kill(pid, SIGKILL);
}
\end{lstlisting}

\section{编译与运行}

\subsection{编译}

程序已集成到 Makefile 中:

\begin{lstlisting}[language=bash]
cd /home/user/UNIX-V6-/src/program
make kill_child.exe
\end{lstlisting}

\subsection{运行}

\begin{lstlisting}[language=bash]
./objs/kill_child.exe
\end{lstlisting}

或在 V6++ 系统中:

\begin{lstlisting}[language=bash]
/zbin1#kill_child
\end{lstlisting}

\section{技术总结}

\subsection{关键知识点}

\begin{enumerate}
    \item \textbf{进程创建}:fork() 系统调用的使用和返回值处理
    \item \textbf{进程通信}:使用信号作为进程间通信机制
    \item \textbf{信号发送}:kill() 系统调用的使用
    \item \textbf{进程回收}:wait() 系统调用防止僵尸进程
    \item \textbf{信号特性}:理解 SIGKILL 的不可捕获性
\end{enumerate}

\subsection{与 immortal 程序的联系}

这两个程序形成了完整的信号处理对比:

\begin{itemize}
    \item \textbf{immortal.c}:演示如何捕获 SIGINT 并拒绝退出
    \item \textbf{kill\_child.c}:演示 SIGKILL 无法被捕获,必定杀死进程
\end{itemize}

这体现了 UNIX 信号机制的两个极端:
\begin{itemize}
    \item \textbf{可捕获信号}:给进程"说不"的权利
    \item \textbf{不可捕获信号}:系统的最终控制权
\end{itemize}

\section{常见问题}

\subsection{Q1: 为什么子进程需要无限循环?}

\textbf{A:} 为了演示子进程在"正常运行"时被强制杀死。如果子进程立即退出,就无法体现 SIGKILL 的作用。

\subsection{Q2: 为什么父进程要 sleep(2)?}

\textbf{A:} 确保子进程已经开始执行。如果立即发送 SIGKILL,可能在子进程还没打印信息时就被杀死。

\subsection{Q3: 为什么一定要调用 wait()?}

\textbf{A:} 回收子进程资源。如果不调用 wait(),子进程会变成僵尸进程(zombie),浪费系统资源。

\subsection{Q4: 可以在子进程中捕获 SIGKILL 吗?}

\textbf{A:} \textbf{不可以}。以下代码无效:

\begin{lstlisting}
signal(SIGKILL, my_handler);  // 系统会忽略这个调用
\end{lstlisting}

\vspace{2em}
\noindent\rule{\textwidth}{0.4pt}

\begin{center}
\textbf{开发环境:} UNIX V6++ 操作系统 \\
\textbf{使用系统调用:} fork(), kill(), wait(), getpid(), sleep(), printf() \\
\textbf{测试状态:} ✓ 通过编译,已集成到构建系统
\end{center}

\end{document}
