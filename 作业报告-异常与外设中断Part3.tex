\documentclass[12pt,a4paper]{ctexart}
\usepackage{geometry}
\usepackage{listings}
\usepackage{xcolor}
\usepackage{graphicx}
\usepackage{hyperref}
\usepackage{enumitem}
\usepackage{tikz}
\usepackage{booktabs}
\usetikzlibrary{shapes.geometric, arrows, positioning, calc}

\geometry{left=2.5cm,right=2.5cm,top=2.5cm,bottom=2.5cm}

% 代码高亮设置
\lstset{
    language=C,
    basicstyle=\ttfamily\small,
    keywordstyle=\color{blue}\bfseries,
    commentstyle=\color{gray}\itshape,
    stringstyle=\color{red},
    numbers=left,
    numberstyle=\tiny\color{gray},
    stepnumber=1,
    numbersep=8pt,
    backgroundcolor=\color{gray!10},
    showspaces=false,
    showstringspaces=false,
    showtabs=false,
    frame=single,
    tabsize=4,
    captionpos=b,
    breaklines=true,
    breakatwhitespace=false,
    escapeinside={\%*}{*)},
    xleftmargin=2em,
    xrightmargin=2em,
    aboveskip=1em,
}

\hypersetup{
    colorlinks=true,
    linkcolor=blue,
    filecolor=magenta,
    urlcolor=cyan,
}

\title{\textbf{作业7 Part 3\\UNIX V6++ 系统异常处理实现\\除零异常与 SIGFPE 信号处理}}
\author{同济大学计算机系 \\ 拔尖班2023}
\date{\today}

\begin{document}

\maketitle
\tableofcontents
\newpage

\section{作业要求}

Part 3 是思考题(总评分加5分),要求在 Unix V6++ 系统中复现除零异常处理程序。具体要求:

\begin{enumerate}[itemsep=0.5em]
    \item 实现一个除法计算程序
    \item 当输入除数为 0 时,能够捕获并处理 SIGFPE 信号
    \item 程序不会因为除零异常而崩溃,而是能够继续运行
    \item 演示异常响应的完整过程
\end{enumerate}

\section{实现方案}

\subsection{信号处理机制分析}

Unix V6++ 系统已经实现了基本的信号处理机制。

\subsubsection{信号定义}

在 \texttt{src/lib/include/sys.h} 中定义了完整的信号集:

\begin{lstlisting}
#define SIGFPE   8    /* Floating point exception */
#define SIGINT   2    /* Interrupt from keyboard */
#define SIGKILL  9    /* Kill(can't be caught or ignored) */
// ... 其他信号定义
\end{lstlisting}

\begin{itemize}
    \item \textbf{SIGFPE}(信号编号 8):浮点异常/除零异常
    \item \textbf{SIGINT}(信号编号 2):键盘中断
    \item \textbf{SIGKILL}(信号编号 9):强制终止等
\end{itemize}

\subsubsection{系统调用}

\begin{lstlisting}
int signal(int signal, void (*func)());
\end{lstlisting}

\begin{itemize}
    \item 用于注册信号处理函数
    \item 在 Unix V6 中,\texttt{signal()} 是\textbf{一次性}的
    \item 必须在处理函数中重新注册
\end{itemize}

\subsubsection{信号处理函数规范}

\begin{itemize}
    \item 必须接受 \texttt{int} 类型参数(信号编号)
    \item 返回值为 \texttt{void}
    \item 需要重新注册以继续捕获同类信号
\end{itemize}

\subsection{程序设计}

创建 \texttt{divzero.c} 程序,实现除零异常处理功能。

\subsubsection{信号处理函数}

\begin{lstlisting}
void sig_dzero(int signo)
{
    if (signo == SIGFPE)
    {
        printf("\n[!] Caught SIGFPE: Division by zero detected!\n");
        printf("Can't divide by zero!\n");
        // 重新注册信号处理函数(Unix V6 特性)
        signal(SIGFPE, sig_dzero);
    }
}
\end{lstlisting}

\subsubsection{主程序流程}

\begin{enumerate}[itemsep=0.5em]
    \item 注册 SIGFPE 信号处理函数
    \item 进入主循环
    \item 接收用户输入的被除数 \texttt{a} 和除数 \texttt{b}
    \item 执行除法运算 \texttt{c = a / b}
    \item 如果 \texttt{b = 0},触发 SIGFPE 信号
    \item 信号处理函数捕获异常并输出错误信息
    \item 程序继续运行,等待下一次输入
\end{enumerate}

\subsubsection{完整代码}

程序位置:\texttt{src/program/divzero.c}

\begin{lstlisting}
#include <stdio.h>
#include <sys.h>

void sig_dzero(int signo)
{
    if (signo == SIGFPE)
    {
        printf("\n[!] Caught SIGFPE: Division by zero detected!\n");
        printf("Can't divide by zero!\n");
        signal(SIGFPE, sig_dzero);
    }
}

int main1(int argc, char* argv[])
{
    int a, b, c;

    if (signal(SIGFPE, sig_dzero) == -1)
    {
        printf("Signal registration failed!\n");
        return -1;
    }

    printf("Division Exception Handler Test Program\n");
    printf("========================================\n");

    while(1)
    {
        printf("Enter dividend a: ");
        scanf("%d", &a);
        printf("Enter divisor b: ");
        scanf("%d", &b);

        if (a == 0 && b == 0)
        {
            printf("Exiting program...\n");
            break;
        }

        c = a / b;  // 除零时触发 SIGFPE

        printf("Result: %d / %d = %d\n\n", a, b, c);
    }

    return 0;
}
\end{lstlisting}

关键特性:
\begin{itemize}
    \item 使用 \texttt{signal(SIGFPE, sig\_dzero)} 注册异常处理函数
    \item 在信号处理函数中重新注册(Unix V6 特性)
    \item 主循环持续运行,不会因异常而退出
    \item 提供友好的用户交互界面
\end{itemize}

\subsection{编译配置}

修改 \texttt{src/program/Makefile},添加 divzero 程序的编译规则:

\begin{enumerate}
    \item 在 \texttt{SHELL\_OBJS} 中添加:
\begin{lstlisting}[language=make]
$(TARGET)\divzero.exe
\end{lstlisting}

    \item 添加编译规则:
\begin{lstlisting}[language=make]
$(TARGET)\divzero.exe : divzero.c
    $(CC) $(CFLAGS) -I"$(INCLUDE)" -I"$(LIB_INCLUDE)" $< -e _main1 $(V6++LIB) -o $@
    copy $(TARGET)\divzero.exe $(MAKEIMAGEPATH)\$(BIN)\divzero
\end{lstlisting}
\end{enumerate}

\section{异常处理过程分析}

\subsection{正常除法流程}

\begin{center}
\begin{tikzpicture}[node distance=1.5cm, auto]
    \tikzstyle{process} = [rectangle, minimum width=3cm, minimum height=1cm, text centered, draw=black, fill=blue!20]
    \tikzstyle{arrow} = [thick,->,>=stealth]

    \node (input) [process] {用户输入: a=10, b=2};
    \node (execute) [process, below of=input] {执行: c = a / b};
    \node (calculate) [process, below of=execute] {计算结果: c = 5};
    \node (output) [process, below of=calculate] {输出结果};
    \node (continue) [process, below of=output] {继续等待输入};

    \draw [arrow] (input) -- (execute);
    \draw [arrow] (execute) -- (calculate);
    \draw [arrow] (calculate) -- (output);
    \draw [arrow] (output) -- (continue);
\end{tikzpicture}
\end{center}

\subsection{除零异常处理流程}

\begin{enumerate}[itemsep=0.3em]
    \item \textbf{用户输入}:a=10, b=0
    \item \textbf{执行除法}:c = a / b
    \item \textbf{CPU 检测}:检测到除零操作
    \item \textbf{触发异常}:硬件触发除零异常(\#DE - Divide Error)
    \item \textbf{保存现场}:
        \begin{itemize}
            \item 将用户态寄存器压入用户栈
            \item 切换到核心态
            \item 将中断返回地址、CS、EFLAGS 等压入核心栈
        \end{itemize}
    \item \textbf{查找处理入口}:通过 IDT(中断描述符表)查找异常处理入口
    \item \textbf{内核处理}:进入内核异常处理函数
    \item \textbf{识别异常}:内核识别为除零异常(Exception \#0)
    \item \textbf{转换信号}:内核将异常转换为 SIGFPE 信号
    \item \textbf{检查处理表}:检查当前进程的信号处理表
    \item \textbf{发现处理函数}:发现用户注册了 SIGFPE 处理函数
    \item \textbf{设置用户栈}:准备调用用户态信号处理函数
    \item \textbf{返回用户态}:从核心态返回用户态
    \item \textbf{执行处理函数}:执行 \texttt{sig\_dzero()}
        \begin{itemize}
            \item 输出错误信息
            \item 重新注册信号处理函数
        \end{itemize}
    \item \textbf{处理函数返回}:信号处理函数返回
    \item \textbf{恢复上下文}:恢复用户程序上下文
    \item \textbf{继续执行}:从触发异常的下一条指令继续执行
    \item \textbf{程序继续}:程序继续运行,等待下一次输入
\end{enumerate}

\subsection{用户栈与核心栈的变化}

\subsubsection{除零异常发生时}

\textbf{用户栈}(用户态):

\begin{center}
\begin{tikzpicture}
    \draw[thick] (0,0) rectangle (4,5);
    \node at (2,4.5) {局部变量 (a, b, c)};
    \draw[thick] (0,4) -- (4,4);
    \node at (2,3.5) {返回地址};
    \draw[thick] (0,3) -- (4,3);
    \node at (2,2.5) {保存的 ebp};
    \draw[thick] (0,2) -- (4,2);
    \node at (2,1.5) {...};
    \draw[->, thick, red] (4.5,2) -- (4.5,2) node[right] {← esp (用户栈顶)};
    \node[above] at (2,5) {高地址};
    \node[below] at (2,0) {低地址};
\end{tikzpicture}
\end{center}

\textbf{核心栈}(切换到核心态后):

\begin{center}
\begin{tikzpicture}
    \draw[thick] (0,0) rectangle (4,6);
    \node at (2,5.5) {SS (用户栈段)};
    \draw[thick] (0,5) -- (4,5);
    \node at (2,4.5) {ESP (用户栈顶)};
    \draw[thick] (0,4) -- (4,4);
    \node at (2,3.5) {EFLAGS};
    \draw[thick] (0,3) -- (4,3);
    \node at (2,2.5) {CS (用户代码段)};
    \draw[thick] (0,2) -- (4,2);
    \node at (2,1.5) {EIP (异常指令)};
    \draw[thick] (0,1) -- (4,1);
    \node at (2,0.5) {Error Code};
    \draw[->, thick, red] (4.5,1) -- (4.5,1) node[right] {← 核心栈顶};
    \node[above] at (2,6) {高地址};
    \node[below] at (2,0) {低地址};
\end{tikzpicture}
\end{center}

\subsubsection{调用信号处理函数时}

\textbf{用户栈}(准备调用信号处理函数):

\begin{center}
\begin{tikzpicture}
    \draw[thick] (0,0) rectangle (4,6);
    \node at (2,5.5) {保存的上下文};
    \draw[thick] (0,5) -- (4,5);
    \node at (2,4.5) {返回地址};
    \draw[thick] (0,4) -- (4,4);
    \node at (2,3.5) {信号编号 (8)};
    \draw[thick] (0,3) -- (4,3);
    \node at (2,2.5) {...};
    \draw[thick] (0,2) -- (4,2);
    \node at (2,1.5) {局部变量 (a, b, c)};
    \draw[->, thick, red] (4.5,3.5) -- (4.5,3.5) node[right] {← 新的 esp};
    \node[above] at (2,6) {高地址};
    \node[below] at (2,0) {低地址};
    \node[left, text width=2.5cm, align=right] at (-0.5,3.5) {信号处理函数参数};
\end{tikzpicture}
\end{center}

\section{与 Ubuntu 系统的对比}

\subsection{相同点}

\begin{enumerate}[itemsep=0.5em]
    \item \textbf{异常触发机制}:都是由 CPU 硬件检测除零操作并触发异常
    \item \textbf{信号转换}:内核都将硬件异常转换为 SIGFPE 信号
    \item \textbf{用户态处理}:都允许用户程序注册信号处理函数
    \item \textbf{栈切换}:都涉及用户栈和核心栈的切换
\end{enumerate}

\subsection{差异点}

\begin{table}[h]
\centering
\begin{tabular}{@{}lll@{}}
\toprule
\textbf{特性} & \textbf{Unix V6++} & \textbf{Ubuntu (现代 Linux)} \\ \midrule
信号注册 & 一次性,需重新注册 & 持久性注册 \\
系统调用 & \texttt{signal()} & \texttt{signal()} 或 \texttt{sigaction()} \\
信号处理函数原型 & \texttt{void func(int)} & \texttt{void func(int)} 或带额外参数 \\
信号掩码 & 简单的信号处理 & 支持复杂的信号掩码和阻塞 \\
可靠性 & 可能丢失信号 & 可靠信号机制 \\
信号队列 & 不支持 & 支持实时信号队列 \\ \bottomrule
\end{tabular}
\caption{Unix V6++ 与 Ubuntu 信号处理机制对比}
\end{table}

\subsection{Unix V6 的特殊性}

\subsubsection{一次性信号处理}

\begin{itemize}
    \item 信号处理函数执行后,自动恢复为默认行为
    \item 必须在处理函数内重新注册
    \item 这是早期 Unix 系统的设计特点
\end{itemize}

\subsubsection{简化的信号机制}

\begin{itemize}
    \item 信号处理相对简单
    \item 没有复杂的信号掩码和阻塞机制
    \item 适合教学和理解基本原理
\end{itemize}

\section{程序使用说明}

\subsection{编译}

在 Unix V6++ 系统中:

\begin{lstlisting}[language=bash]
cd src/program
make divzero.exe
\end{lstlisting}

\subsection{运行示例}

\begin{lstlisting}[language=bash]
> divzero

Division Exception Handler Test Program
========================================
This program demonstrates SIGFPE signal handling.
Enter two integers for division. Enter 0 0 to exit.

Enter dividend a: 10
Enter divisor b: 2
Result: 10 / 2 = 5

Enter dividend a: 10
Enter divisor b: 0

[!] Caught SIGFPE: Division by zero detected!
Can't divide by zero!

Enter dividend a: 8
Enter divisor b: 4
Result: 8 / 4 = 2

Enter dividend a: 0
Enter divisor b: 0
Exiting program...
\end{lstlisting}

\subsection{测试要点}

\begin{enumerate}[itemsep=0.5em]
    \item \textbf{正常除法}:输入正常的被除数和除数,验证计算正确
    \item \textbf{除零异常}:输入除数为 0,验证异常被捕获
    \item \textbf{多次除零}:多次输入除数为 0,验证信号重新注册成功
    \item \textbf{程序持续性}:验证程序在异常后能继续运行
    \item \textbf{退出功能}:输入 0 0 正常退出程序
\end{enumerate}

\section{技术要点总结}

\subsection{信号处理的关键步骤}

\begin{enumerate}[itemsep=0.5em]
    \item \textbf{注册信号处理函数}:
    \begin{lstlisting}
signal(SIGFPE, sig_dzero);
    \end{lstlisting}

    \item \textbf{实现信号处理函数}:
    \begin{lstlisting}
void sig_dzero(int signo) {
    // 处理异常
    // 重新注册(Unix V6 特性)
    signal(SIGFPE, sig_dzero);
}
    \end{lstlisting}

    \item \textbf{触发异常}:
    \begin{lstlisting}
c = a / b;  // 当 b = 0 时触发 SIGFPE
    \end{lstlisting}
\end{enumerate}

\subsection{Unix V6++ 信号机制特点}

\begin{itemize}[itemsep=0.5em]
    \item \textbf{一次性注册}:与现代 Linux 的持久性注册不同
    \item \textbf{必须重新注册}:在处理函数中调用 \texttt{signal()} 重新注册
    \item \textbf{简洁的 API}:只有基本的 \texttt{signal()} 和 \texttt{kill()} 系统调用
    \item \textbf{教学价值}:清晰展示信号处理的基本原理
\end{itemize}

\subsection{异常处理的意义}

\begin{enumerate}[itemsep=0.5em]
    \item \textbf{程序健壮性}:避免因异常输入导致程序崩溃
    \item \textbf{用户体验}:提供友好的错误提示
    \item \textbf{系统稳定性}:防止单个进程异常影响整个系统
    \item \textbf{错误恢复}:允许程序从错误状态恢复并继续运行
\end{enumerate}

\section{实验收获}

\subsection{深入理解异常处理机制}

\begin{itemize}[itemsep=0.5em]
    \item 从硬件异常触发到软件信号处理的完整流程
    \item 用户态和核心态的切换过程
    \item 栈的切换和上下文保存恢复
\end{itemize}

\subsection{Unix V6 系统特性}

\begin{itemize}[itemsep=0.5em]
    \item 一次性信号注册机制
    \item 简洁的系统调用接口
    \item 与现代 Linux 的异同
\end{itemize}

\subsection{实践能力提升}

\begin{itemize}[itemsep=0.5em]
    \item 在真实操作系统中实现信号处理
    \item 理解系统调用的使用方法
    \item 掌握异常处理的编程技巧
\end{itemize}

\subsection{系统设计思想}

\begin{itemize}[itemsep=0.5em]
    \item \textbf{分层设计}:硬件异常 → 内核处理 → 用户处理
    \item \textbf{灵活性}:允许用户自定义异常处理行为
    \item \textbf{简洁性}:Unix V6 的设计哲学
\end{itemize}

\section{参考资料}

\begin{enumerate}[itemsep=0.5em]
    \item Unix V6++ 源代码:\texttt{src/lib/include/sys.h}
    \item 信号处理示例:\texttt{src/program/immortal.c}, \texttt{src/program/sigTest.c}
    \item 操作系统课程讲义:异常与中断处理
    \item UNIX V6 源代码注释
\end{enumerate}

\end{document}
